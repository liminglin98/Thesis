\documentclass[12pt]{article}
\usepackage[margin=1in]{geometry}
\usepackage{amsmath, amssymb}
\usepackage{setspace}
\usepackage{enumitem}
\usepackage{hyperref}

\title{Thesis Proposal: \textit{NGDP Targeting under Fiscal Leadership: A Framework for China's Monetary-Fiscal Interaction}}
\author{Liming Lin \\ Sciences Po Paris}
\date{May 2025}

\begin{document}

\maketitle
\onehalfspacing

\section*{Motivation}
China's monetary policy is increasingly constrained: interest rates are near the zero lower bound (ZLB), and the People's Bank of China (PBoC) lacks full independence. Meanwhile, fiscal policy plays a dominant role in demand management. Traditional inflation targeting may no longer provide a stable nominal anchor. This project explores whether \textbf{nominal GDP (NGDP) targeting} can serve as a more effective framework under these conditions.

\section*{Research Questions}
\begin{itemize}[noitemsep]
  \item How does NGDP targeting compare to inflation targeting in a fiscal-dominant, ZLB-constrained environment?
  \item Can NGDP targeting improve macroeconomic stability and sectoral balance in China?
  \item Can NGDP targeting serve as a coordination device between monetary and fiscal authorities?
\end{itemize}

\section*{Model Framework}
\textbf{Stage 1:} RANK (Representative Agent New Keynesian) model with sticky prices and ZLB constraint. Includes:
\begin{itemize}[noitemsep]
  \item Monetary policy: Taylor rule vs. NGDP targeting rule
  \item Fiscal policy: spending rule responding to output or debt
\end{itemize}
\textbf{Stage 2:} Add game-theoretic structure: PBoC and MoF minimize separate loss functions.
\begin{itemize}[noitemsep]
  \item Stackelberg game (MoF leads), or Nash game (simultaneous moves)
\end{itemize}
\textbf{Stage 3:} Introduce sectoral heterogeneity (e.g., real estate, manufacturing) in production side for policy transmission analysis.

\section*{Empirical Strategy (Optional)}
Use macro and sector-level data (e.g., NBS, IMF, CEIC) to:
\begin{itemize}[noitemsep]
  \item Calibrate sectoral elasticities to monetary/fiscal shocks
  \item Estimate reaction functions of policy instruments to NGDP or inflation
\end{itemize}

\section*{Contribution}
\begin{itemize}[noitemsep]
  \item Develops a theoretical framework reflecting China's institutional context
  \item Provides normative case for NGDP targeting under fiscal dominance
  \item Offers a foundation for sectoral and empirical extensions
\end{itemize}

\end{document}