\documentclass{beamer}
\usepackage[normalem]{ulem}
\usetheme{Madrid}

\title{NGDP Targeting under Fiscal Leadership: A Framework for China's Monetary-Fiscal Interaction}
\author{Liming Lin}
\institute{Sciences Po Paris}
\date{May 2025}

\begin{document}

\begin{frame}
  \titlepage
\end{frame}

% Slide 1: Motivation
\begin{frame}{Motivation}
  \begin{itemize}
    \item \textbf{Stagnant demand}: GDP growth slowing, inflation near 0\%, interest rate at 1.4\%—close to ZLB
    \begin{itemize}
     \item \textbf{Demand-driven slowdown}: Current weakness reflects demand shocks, not supply constraints
    \end{itemize}

    \item \textbf{No explicit policy rule}: PBoC lacks a clear target but the State Council has annual GDP growth rate targets
    \item \textbf{Weak coordination}: PBoC has limited independence; MoF remains passive during downturns
    \item \textbf{NGDP targeting as a solution}: Provides a rule-based anchor and coordinates fiscal-monetary policy
  \end{itemize}
\end{frame}

\begin{frame}
\frametitle{NGDP Targeting: Key Literature Insights}

\textbf{1. Critiques of Traditional Rules}

\begin{itemize}
  \item \textbf{Orphanides (2001-2003)}:
  \begin{itemize}
    \item Real-time data on output gap and $r^*$ is noisy.
    \item Taylor rule performance deteriorates with mismeasurement.
  \end{itemize}

  \item \textbf{McCallum (2000)}:
  \begin{itemize}
    \item Interest-rate rules rely on latent natural rate assumptions.
    \item In unstable environments, quantity-based rules are more robust.
    \item Calls for return to monetarist anchors like base growth or NGDP paths.
  \end{itemize}
\end{itemize}

\vspace{0.3cm}
\textbf{2. NGDP Targeting: Theoretical and Empirical Support}

\begin{itemize}
  \item \textbf{Garín, Lester \& Sims (2016):}
  \begin{itemize}
    \item In NK model with sticky wages/prices, NGDP targeting $\approx$ output gap targeting in welfare.
    \item Requires no unobservable gap estimates.
  \end{itemize}

  \item \textbf{Beckworth \& Hendrickson (2019):}
  \begin{itemize}
    \item NGDP targeting linked to lower inflation volatility historically.
  \end{itemize}
\end{itemize}

\end{frame}

\begin{frame}
\frametitle{Policy Rules: Formulas and Comparison}

\textbf{1. McCallum Rule (Quantity-based)}:
\[
\Delta b_t = \Delta x^* + \gamma (\Delta x^* - \Delta x_{t-1})
\]

Targets growth of monetary base to stabilize NGDP growth.

\textbf{2. Taylor Rule (Interest-rate-based)}:
\[
i_t = r^* + \pi_t + \phi_\pi (\pi_t - \pi^*) + \phi_y (y_t - y^*)
\]

Responds to inflation and output gaps—both difficult to estimate in real time.

\textbf{3. NGDP Targeting}:
\[
ln(1+r_t)=\phi_\pi(1+\pi_t)+\phi_y ln (\frac{Y_t}{Y_{t-1}}) \text{(Garín, Lester \& Sims(2016))}
\]

\[
r_t=\rho r_{t-1}+(1-\rho)\Omega\Delta x_t +e_t^r \text{(Beckworth \& Hendrickson (2019))}
\]



\end{frame}

\begin{frame}{\sout{Potential Research Questions 1}}
  \begin{itemize}
    \item Does an NGDP Targeting policy rule explain PBoC behavior better than traditional inflation-targeting rules?
    \item Past papers (Burdekin \& Siklos (2024), Girardin, Lunven \& Ma (2017)) have shown that PBoC's policy rule is anti-inflation and follows both Taylor and McCallum rules to some extent.
    \item \textbf{Contribution}: \textcolor{red}{Estimating a NGDP targeting rule using GMM or ? to see if it fits better than Taylor or McCallum rules. Maybe add a VAR model for impulse response analysis.}
  \end{itemize}
\end{frame}

\begin{frame}{Potential Research Questions 2}
  
    \begin{itemize}
      \item Can NGDP targeting outperform inflation targeting and coordinate stabilization efforts between the PBoC and MoF in a setting where one or both face institutional or economic constraints?
      \item Beckworth \& Hendrickson (2019) incoporated NGDP targeting into a New Keynesian model found that it creates lower volatility in inflation than Taylor-type rules.
      \item \textbf{Contribution}: RANK model with ZLB constraint using Chinese data
      \begin{itemize}
        \item Monetary rule: Taylor vs. NGDP targeting
        \item Fiscal rule: Exogenous (Simple AR(1)) vs. \textcolor{red}{Output gap -reactive (like Taylor rule) vs. NGDP targeting}
      \end{itemize}
    \end{itemize}

\end{frame}

\begin{frame}{Model Setup}
  \begin{itemize}
    \item Monetary rule: Taylor vs. \textcolor{red}{NGDP targeting}
      \begin{itemize}
        \item Level Targeting: $r_t=\phi_x(x_t-x^*)$
        \item Growth Targeting: $r_t=\phi_x(\Delta x_t-\Delta x^*)$
        \item Can also add other components like AR(1)
      \end{itemize}
    \item Fiscal rule: Exogenous (Simple AR(1)) + Steady Level Spending vs. \textcolor{red}{Output gap -reactive (like Taylor rule) vs. NGDP targeting}
      \begin{itemize}
        \item Output gap reactive: $g_t=\gamma_y(y_t-y^f)$
        \item NGDP Level Targeting: $g_t=\gamma_x(x_t-x^*)$
        \item NGDP Growth Targeting: $g_t=\gamma_x(\Delta x_t-\Delta x^*)$
      \end{itemize}
  \end{itemize}
\end{frame}

\begin{frame}{Monetary-Fiscal Interaction in China}
\begin{itemize}
    \item In China, fiscal and monetary authorities (MoF and PBoC) interact strategically.
    \item Recent events (e.g., 1.1 trillion RMB profit remittance) highlight blurred institutional boundaries.
    \item Objective: model these interactions using game theory to understand outcomes under NGDP targeting.
\end{itemize}
\end{frame}

\begin{frame}{Leeper (1991): Active vs. Passive Policy}
\begin{itemize}
    \item Defines monetary and fiscal regimes as \textbf{active} or \textbf{passive}.
    \item \textbf{Active monetary policy:} strong response to inflation
    \item \textbf{Passive fiscal policy:} adjusts to ensure intertemporal budget constraint holds
    \item Uniqueness of equilibrium depends on the mix:
    \begin{itemize}
        \item (Monetary active, fiscal passive) $\Rightarrow$stable prices
        \item (Both active/passive) $\Rightarrow$ possible indeterminacy
    \end{itemize}
\end{itemize}
\end{frame}

\begin{frame}{Dixit and Lambertini (2003)}
\begin{itemize}
    \item Monetary and fiscal authorities minimize separate loss functions:
\end{itemize}
\[
    L_{\text{CB}} = \frac{1}{2}(\pi - \pi^*)^2 + \frac{\lambda}{2}y^2
\]
\[
    L_{\text{F}} = \frac{1}{2}(\pi - \pi^*)^2 + \frac{\beta}{2}y^2 + \frac{\gamma}{2}g^2
\]
\begin{itemize}
    \item Economy described by:
\end{itemize}
\[
    y = a g - b r
\]
\[
    \pi = \kappa y
\]
\begin{itemize}
    \item Equilibrium depends on timing:
    \begin{itemize}
        \item Nash: each authority ignores externalities $\Rightarrow$ inflation bias
    \end{itemize}
\end{itemize}
\end{frame}

\begin{frame}{My Setup: Stackelberg Game (MoF Leads)}
\begin{itemize}
    \item Fiscal authority maximizes:
    \[ U_{\text{MoF}} = y - \frac{\gamma}{2}g^2 - \frac{\theta}{2}\pi^2 \]
    \item PBoC minimizes:
    \[ L_{\text{PBoC}} = \frac{1}{2}\pi^2 + \frac{\lambda}{2}y^2 \]
    \item Subject to:
    \[ y = a g - b r, \quad \pi = \kappa y \]
\end{itemize}
\begin{itemize}
    \item Solve by backward induction:
    \begin{enumerate}
        \item PBoC best response $\Rightarrow$ $r(g)$
        \item MoF chooses $g$ anticipating $r(g)$
    \end{enumerate}
\end{itemize}
\end{frame}

\begin{frame}{Equilibrium Intuition}
\begin{itemize}
    \item Under \textbf{fiscal leadership}, monetary policy accommodates spending $\Rightarrow$ inflation bias
    \item Chinese context: MoF may effectively lead due to treasury control, land finance, local incentives
\end{itemize}
\end{frame}

\begin{frame}{NGDP Targeting Variant (Placeholder)}
\begin{itemize}
    \item Under NGDP targeting, both authorities aim to stabilize nominal income:
\end{itemize}
\begin{align*}
    L_{\text{PBoC}} &= \phi_{\text{pboc}} (y - y^*)^2 \\
    U_{\text{MoF}} &= \phi_{\text{mof}} (y - y^*)^2
\end{align*}
\begin{itemize}
    \item Extensions can consider how fiscal leadership affects NGDP-targeting credibility
\end{itemize}
\end{frame}

\begin{frame}{Possible Extensions}
\begin{itemize}
    \item Add local governments as strategic agents:
    \begin{itemize}
        \item Nodes in a network
        \item Invest in real estate vs. productive sectors
    \end{itemize}
    \item Explain why stimulus flows disproportionately into real estate
    \item Link game outcome to fiscal rule in RANK model with NGDP targeting
\end{itemize}
\end{frame}


\end{document}
