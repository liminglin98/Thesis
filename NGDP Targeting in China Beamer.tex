\documentclass{beamer}
\usetheme{Madrid}

\title{NGDP Targeting under Fiscal Leadership: A Framework for China's Monetary-Fiscal Interaction}
\author{Liming Lin}
\institute{Sciences Po Paris}
\date{May 2025}

\begin{document}

\begin{frame}
  \titlepage
\end{frame}

% Slide 1: Motivation
\begin{frame}{Motivation}
  \begin{itemize}
    \item \textbf{Stagnant demand}: GDP growth slowing, inflation near 0\%, interest rate at 1.4\%—close to ZLB
    \begin{itemize}
     \item \textbf{Demand-driven slowdown}: Current weakness reflects demand shocks, not supply constraints
    \end{itemize}

    \item \textbf{No explicit policy rule}: PBoC lacks a clear target but the State Council has annual GDP growth rate targets
    \item \textbf{Weak coordination}: PBoC has limited independence; MoF remains passive during downturns
    \item \textbf{NGDP targeting as a solution}: Provides a rule-based anchor and coordinates fiscal-monetary policy
  \end{itemize}
\end{frame}

% Slide 2: Research Questions
\begin{frame}{Potential Research Questions 1}
  \begin{itemize}
    \item Does an NGDP Targeting policy rule explain PBoC behavior better than traditional inflation-targeting rules?
    \item Past papers (Burdekin \& Siklos (2024), Girardin, Lunven \& Ma (2017)) have shown that PBoC's policy rule is anti-inflation and follows both Taylor and McCallum rules to some extent.
    \item \textbf{Contribution}: \textcolor{red}{Estimating a NGDP targeting rule using GMM or ? to see if it fits better than Taylor or McCallum rules. Maybe add a VAR model for impulse response analysis.}
  \end{itemize}
\end{frame}

% Slide 3: Model Overview
\begin{frame}{Potential Research Questions 2}
  \begin{itemize}
    \item Can NGDP targeting outperform inflation targeting and coordinate stabilization efforts between the PBoC and MoF in a setting where one or both face institutional or economic constraints?
    \item Beckworth \& Hendrickson (2020) incoporated NGDP targeting into a New Keynesian model found that it creates lower volatility in inflation than Taylor-type rules.

    \item \textbf{Contribution}: RANK model with ZLB constraint using Chinese data
    \begin{itemize}
      \item Monetary rule: Taylor vs. NGDP targeting
      \item Fiscal rule: Exogenous (Simple AR(1)) vs. \textcolor{red}{Output gap -reactive (like Taylor rule) vs. NGDP targeting}
    \end{itemize}
  \end{itemize}
\end{frame}


% Slide 5: Contribution and Next Steps
\begin{frame}{Strategic Interactions between PBoC and MoF ?}
  \begin{itemize}
    \item \textbf{Stage 2:} Add policy interaction between PBoC and MoF (Stackelberg/Nash)
    \begin{itemize}
      \item Monetary and fiscal authorities modeled with separate rules:
      \begin{itemize}
        \item PBoC: interest rate responds to inflation/output gap/NGDP
        \item MoF: government spending responds to debt/output gap/NGDP
      \end{itemize}
      \item Strategic setups:
      \begin{itemize}
        \item \textbf{Stackelberg:} MoF leads, PBoC follows
        \item \textbf{Nash:} Simultaneous policy moves
      \end{itemize}
      \item NGDP targeting tested as \textbf{coordination anchor}
    \end{itemize}
  \end{itemize}
\end{frame}

\end{document}
